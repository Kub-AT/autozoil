\documentclass[skorowidz, autorrok, backref, xodstep]{wmimgr}
%\pagestyle{headings}

% Opcjonalnie identyfikator dokumentu (drukowany tylko z włączoną opcją `brudnopis')
\nrwersji {0.1}

\author   {Roman Grundkiewicz}
\nralbumu {316826}
\email    {rg33@st.amu.edu.pl}

\title    {Automatyczne wyszukiwanie i~grupowanie krótkich tekstów narracyjnych zamieszczanych w~Internecie}
\kierunek {Informatyka}
\date     {2011}
\miejsce  {Poznań}
\opiekun  {dr Filip Graliński}

%
% Miejsce na deklaracje własnych poleceń:
\newcommand{\filename}[1]{\texttt{#1}}

\newcommand{\bs}{$\backslash$} 
\newcommand{\eng}[1]{(ang. \emph{#1})}
\newcommand{\refsee}[2][]{(zob. #1 \ref{#2})}
\newcommand{\refcomp}[2][]{(por. #1 \ref{#2})}
\newcommand{\mypicture}[4][0.2]{
	\begin{figure}[!htb]
		\begin{center}
		\includegraphics[scale=#1]{img/#2}
		\caption[#3]{\label{#2}\textit{#4}}
		\end{center}
	\end{figure}
}
\mathchardef\mhyphen="2D
\newcommand{\tfidf}{$\mathit{ tf \mhyphen idf_{t,d} }$}
\newcommand{\tfidfraw}{$\mathit{ tf \mhyphen idf }$}

\usepackage{algorithm, algorithmic}
\usepackage{color}
\usepackage{todo}

\definecolor{myred}{RGB}{233,93,15}
\definecolor{mygreen}{RGB}{151,191,13}
\definecolor{myblue}{RGB}{0,158,224}
\definecolor{mygray}{gray}{0.75}

\floatname{algorithm}{Algorytm}
\renewcommand{\listalgorithmname}{Spis algorytmów} 
\renewcommand{\algorithmicrequire}{\textbf{Input:}}
\renewcommand{\algorithmicensure}{\textbf{Output:}}
\renewcommand{\algorithmicforall}{\textbf{foreach}}

% Cytowanie przez numer (standard):
%\bibliographystyle{plain} 
% Jeżeli cytowanie autor-rok to np.:
%\bibliographystyle{papalike}
% Inne sposoby
\bibliographystyle{abbrv}

\begin{document}

%\nocite{Tolkien2011} %% dołącza niecytowany tekst

\begin{abstract}
  Tutaj będzie streszczenie.
\end{abstract}
\keywords{grupowanie, klasteryzacja, wyszukiwanie informacji, k-średnich, wybór centroidów, 
	przetwarzanie tekstu, teksty narracyjne, legendy miejskie}
	
% TODO: dodać angielskie keywords:
% {documents clustering, information retrieval, k-means, seeds, narrative texts, urban legends} 

\maketitle

\introduction
Wstęp z~opisem celu i~zakresu pracy.

\chapter{Przegląd zagadnień}
	\section{Wyszukiwanie informacji}
	\section{Charakterystyka tekstów narracyjnych}
	\section{Grupowanie dokumentów}
	
% grupowanie wyników wyszukiwania (Manning2009, Weiss2001)
	
		\subsection{Systemy grupujące}
		\subsubsection{Carrot}
	
\chapter{Reprezentacja dokumentów w~przestrzeniach wielowymiarowych}

Aby móc grupować dokumenty tekstowe, konieczne jest zastosowanie takiej ich komputerowej reprezentacji, 
która będzie umożliwiała automatyczne ich porównywanie.

W niniejszym rozdziale szczegółowo opisano model wektorowy oraz techniki budowy bazy przestrzeni cech. 
Następnie zestawione zostały najpopularniejsze konstrukcje miar odległości w przestrzeniach wielowymiarowych.
Krótko przedstawiono również alternatywne sposoby reprezentacji dokumentów. 

	\section{Model wektorowy}
	\label{model-wektorowy}
	
Najpopularniejszym sposobem reprezentacji dokumentów jest model wektorowy \eng{VSM, Vector Space Model}
\cite{Broda2007, Manning2009, Kosmulski2005}.
